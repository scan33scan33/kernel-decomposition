%\documentclass[14pt,xcolor=pst]{beamer}
\documentclass[14pt,xcolor=pstdvips]{beamer}
\usetheme{Madrid}
%\mode<presentation>{\usetheme{Warsaw}}
%,/usr/faculty/professor/cjlin/latex/labelfig}
% \pagecolor{yellow}
\def\printlandscape{\special{landscape}}    % Works with dvips.
\setbeamercolor{alerted text}{fg=red!80!yellow}
%\setbeamercolor{alerted text}{fg=red!80!yellow}


%\usepackage{labelfig}
%\usepackage[round,authoryear]{natbib}
\usepackage{array,amsmath,latexsym,epsfig,float,afterpage,alltt,amssymb,theorem,enumerate,pstricks,bm,hhline,url,xr,float,multicol,multirow,ragged2e,colortbl,xspace, natbib}
\usepackage{ifthen}
\usepackage{tikz}

%\usepackage{CJKutf8} % since thesis.tex uses CJK
%\usepackage[
%  unicode,
%  bookmarksnumbered=true,
%  breaklinks=true,
%]{hyperref}
%\hypersetup{
%  pdfauthor={\myCname (\myEname)},
%  pdftitle={\cTitle (\eTitle)},
%}

%\externaldocument{../thesis} % xref

\usepackage{bibentry}

\usetikzlibrary{calc,trees,positioning,arrows,chains,shapes.geometric,%
    decorations.pathreplacing,decorations.pathmorphing,shapes,%
    matrix,shapes.symbols}

\tikzset{
  >=stealth',
  punktchain/.style={
    rectangle, 
    rounded corners, 
    draw=black, thick,
    text width=10em, 
    minimum height=1em, 
    text centered, 
    on chain},
  line/.style={draw, thick, <-},
  every join/.style={->, thick,shorten >=1pt},
}



%\input{../macro} % pull pre-defined macros
%\input{../info} % pull pre-defined macros

%\newboolean{thesis}
%\newcommand{\best}{\alert}

%\AtBeginSubsection[]
%{
%    \begin{frame}<beamer>
%        \frametitle{Outline}
%        \tableofcontents[current,currentsubsection]
%    \end{frame}
%}

\def\bw{{\boldsymbol w}}
\def\bx{{\boldsymbol x}}
\def\by{{\boldsymbol y}}
\def\bxi{{\boldsymbol \xi}}
\def\svmlin{{\sf SVMlin}\xspace}
\def\libsvm{{\sf LIBSVM}\xspace}
\def\liblinear{{\sf LIBLINEAR}\xspace}
\def\vw{{\sf VW}\xspace}
\def\HIVA{{\sf HIVA}\xspace}
\def\IBNSINA{{\sf IBN\_SINA}\xspace}
\def\NOVA{{\sf NOVA}\xspace}
\def\ORANGE{{\sf ORANGE}\xspace}
\def\SYLVA{{\sf SYLVA}\xspace}
\def\SYLVAexp{{\sf SYLVAexp}\xspace}
\def\ZEBRA{{\sf ZEBRA}\xspace}
\def\A{{\sf A}\xspace}
\def\B{{\sf B}\xspace}
\def\C{{\sf C}\xspace}
\def\D{{\sf D}\xspace}
\def\E{{\sf E}\xspace}
\def\F{{\sf F}\xspace}




\begin{document}


\title[Nystr\"om Decomposition]{Solving Non-Linear SVM in Linear Time -- A Nystr\"om Approximated SVM with Applications to Image Classification}
\author[Ming-Hen Tsai]{
Ming-Hen Tsai\\
Academia Sinica\\ 
(now at Google Inc.)\\
\smallskip
\smallskip
Joint work with Yi-Ren Yeh, Yuh-Jye Lee and Yu-Chiang Wang \\
}
\institute[Academia Sinica]{}
\date[May 24, 2013]{}


\begin{frame}
  \titlepage
\end{frame}

\begin{frame}
  \frametitle{Outline}
  \tableofcontents[pausesections]
\end{frame}


\section{ }
\subsection{Introduction}
\begin{frame}
  \frametitle{Outline}
  \tableofcontents[current]
\end{frame}

\begin{frame}
  \frametitle{We Have Known..}
  \begin{itemize}
    \item Non-linear SVM: powerful but slow 
    \item Linear SVM: simple but fast  
    \item [] Paper: "Training Linear SVMs in Linear Time" by Joachims
    \item [] Software: \liblinear, \vw, etc.
 \end{itemize}
\end{frame}

\begin{frame}
  \frametitle{We Had Always Wondered...}
  \begin{itemize}
    \item But we want something {\bf powerful} and {\bf fast}: train faster than non-linear SVM and generate a more accurate model than linear SVM.
    \pause
    \item [] What should we do? 
    \pause
    \item Can we save some computations (for speed) and still get non-linearity (for more power)?
    \pause
    \item [] Yes. Do approximation!
  \end{itemize}
\end{frame}


\subsection{Nystr\"om Method for Kernel Approximation}
\begin{frame}
  \frametitle{Kernel in Non-linear SVM}
\end{frame}

\begin{frame}
  \frametitle{Nystr\"om Method}
\end{frame}

\subsection{Nystr\"om Method for Linear Representation}
\begin{frame}
  \frametitle{An Equivalent Representation}
\end{frame}

\begin{frame}
  \frametitle{The Equivalent Representation in Nystr\"om Method}
\end{frame}

\subsection{Experiment}

\subsection{Conclusion and Application Areas}
\begin{frame}
  \frametitle{Conclusion and Future Work}
\end{frame}

\end{document}
